La Universidad del Valle de Guatemala cuenta con un laboratorio de Robótica llamado Robotat,  el cual cuenta con un sistema de captura de movimiento y se trabajan líneas de investigación relacionadas con sistemas de control y robótica, además de ser el espacio donde se realizan las prácticas de laboratorio de los cursos de robótica. Está diseñado para realizar pruebas con agentes autónomos, entre ellos robots humanoides, robots móviles con ruedas, brazos robóticos y drones.

Entre los drones a utilizar se encuentran los drones Crazyflie 2.1, los cuales son de tamaño pequeño y pueden emplearse en conjunto para controlar enjambres de drones. Se planea realizar este control mediante el sistema Crazyswarm, el cual funciona a trvés de ROS2 en Linux. Con el objetivo de que este sistema de drones pueda utilizarse en prácticas de laboratorio y otras líneas de investigación en la Universidad, se adaptará la infraestructura de este sistema al ecosistema Robotat a través de los paquetes y funciones de ROS2. Esto podrá realizarse mediante la obtención de paquetes de información generados por los algoritmos de control y el sistema de captura de movimiento del Robotat, logrando una integración entre los drones y el laboratorio.