Los drones son dispositivos que han ganado popularidad en los últimos años por una característica especial, ya que al modelarlos como un sistema dinámico se presenta una dinámica no lineal, pero pueden ser controlados de forma satisfactoria con un controlador lineal, siendo este un control relativamente sencillo. Esto convierte a los drones en un excelente método de aprendizaje para diseñar y analizar modelos y algoritmos de control. Debido a esto, trabajar físicamente con un drone o inclusive, con un conjunto de estos, es una alternativa para poner en práctica los aprendizajes adquiridos a lo largo de una carrera universitaria relacionada con electrónica y para realizar investigación relaciona con sistemas de control clásico, moderno y robótica de enjambre. Para poder realizar esto, es necesario contar con un medio que le provea a los algoritmos de control la información necesaria para trabajar, incluyendo, pero no limitándose, a las posiciones espaciales y orientaciones de los respectivos drones. Debido a que se cuenta con un entorno que es capaz de capturar esta información física, el camino a tomar es la creación de un intermediario que permita la comunicación entre el sistema de captura y los algoritmos de control para drones.

El ecosistema Robotat fue creado como un entorno tecnológico que sirviera como un lugar de aprendizaje de sistemas robóticos y para desarrollar lineas de investigación relacionadas con sistemas de control y robótica. Para culminar el desarrollo de este entorno, es necesario integrar a todos los agentes autónomos que formarán parte de él, siendo los drones Crazyflie una parte importante del ecosistema. 